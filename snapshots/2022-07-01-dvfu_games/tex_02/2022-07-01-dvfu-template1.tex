% arara: xelatex: {shell: yes}
%% arara: biber
%% arara: xelatex: {shell: yes}
%% arara: xelatex: {shell: yes}
\documentclass[12pt]{article}

\usepackage{etoolbox} % для условия if-else
\newtoggle{excerpt} % помечаем, что это отрывок, а далее в тексте может использовать
\toggletrue{excerpt}
% команду \iftoggle{excerpt}{yes}{no}

\usepackage{tikz} % картинки в tikz
\usepackage{microtype} % свешивание пунктуации

\usepackage{array} % для столбцов фиксированной ширины

\usepackage{indentfirst} % отступ в первом параграфе

\usepackage{sectsty} % для центрирования названий частей
\allsectionsfont{\centering}

\usepackage{amsmath} % куча стандартных математических плюшек

\usepackage{comment}
\usepackage{amsfonts}

\usepackage[top=2cm, left=1.2cm, right=1.2cm, bottom=2cm]{geometry} % размер текста на странице

\usepackage{lastpage} % чтобы узнать номер последней страницы

\usepackage{enumitem} % дополнительные плюшки для списков
%  например \begin{enumerate}[resume] позволяет продолжить нумерацию в новом списке
\usepackage{caption}


\usepackage{fancyhdr} % весёлые колонтитулы
\pagestyle{fancy}
\lhead{Теория игр — ДВФУ}
\chead{}
\rhead{Финальный экзамен, 2022-07-01}
\lfoot{}
\cfoot{}
\rfoot{\thepage/\pageref{LastPage}}
\renewcommand{\headrulewidth}{0.4pt}
\renewcommand{\footrulewidth}{0.4pt}



\usepackage{todonotes} % для вставки в документ заметок о том, что осталось сделать
% \todo{Здесь надо коэффициенты исправить}
% \missingfigure{Здесь будет Последний день Помпеи}
% \listoftodos --- печатает все поставленные \todo'шки


% более красивые таблицы
\usepackage{booktabs}
% заповеди из докупентации:
% 1. Не используйте вертикальные линни
% 2. Не используйте двойные линии
% 3. Единицы измерения - в шапку таблицы
% 4. Не сокращайте .1 вместо 0.1
% 5. Повторяющееся значение повторяйте, а не говорите "то же"

\usepackage{multicol}


\usepackage{fontspec}
\usepackage{polyglossia}

\setmainlanguage{russian}
\setotherlanguages{english}

% download "Linux Libertine" fonts:
% http://www.linuxlibertine.org/index.php?id=91&L=1
\setmainfont{Linux Libertine O} % or Helvetica, Arial, Cambria
% why do we need \newfontfamily:
% http://tex.stackexchange.com/questions/91507/
\newfontfamily{\cyrillicfonttt}{Linux Libertine O}

\AddEnumerateCounter{\asbuk}{\russian@alph}{щ} % для списков с русскими буквами
\setlist[enumerate, 2]{label=\alph*),ref=\alph*} % \alph* \asbuk* \arabic*

%% эконометрические сокращения
\DeclareMathOperator{\Cov}{Cov}
\DeclareMathOperator{\Corr}{Corr}
\DeclareMathOperator{\Var}{Var}
\DeclareMathOperator{\E}{E}
\newcommand \hb{\hat{\beta}}
\newcommand \hs{\hat{\sigma}}
\newcommand \htheta{\hat{\theta}}
\newcommand \s{\sigma}
\newcommand \hy{\hat{y}}
\newcommand \hY{\hat{Y}}
\newcommand \e{\varepsilon}
\newcommand \he{\hat{\e}}
\newcommand \z{z}
\newcommand \hVar{\widehat{\Var}}
\newcommand \hCorr{\widehat{\Corr}}
\newcommand \hCov{\widehat{\Cov}}
\newcommand \cN{\mathcal{N}}
\let\P\relax
\DeclareMathOperator{\P}{\mathbb{P}}

\DeclareMathOperator{\plim}{plim}



\usepackage{color,url,amsthm,amssymb,longtable,eurosym}
\newenvironment{question}{\item }{}
% \newenvironment{solution}{}{}
\excludecomment{solution}
\newenvironment{answerlist}{\begin{multicols}{3}\begin{enumerate}}{\end{enumerate}\end{multicols}}
\newcommand{\setzerocols}{
  \renewenvironment{answerlist}{\begin{enumerate}}{\end{enumerate}}
}
\newcommand{\setncols}[1]{
  \renewenvironment{answerlist}{\begin{multicols}{#1}\begin{enumerate}}{\end{enumerate}\end{multicols}}
}


\newcommand{\answerbox}{\raisebox{3mm}{%
    \fbox{%
          \begin{minipage}[t]{2mm}%
              \hspace*{2mm}%
              \vspace*{0.3cm}
          \end{minipage}
         }
    }
}

\newcommand{\answerline}{
    \answerbox\hspace*{-1mm}a \hspace*{3mm}
    \answerbox\hspace*{-1mm}b \hspace*{3mm}
    \answerbox\hspace*{-1mm}c \hspace*{3mm}
    \answerbox\hspace*{-1mm}d \hspace*{3mm}
    \answerbox\hspace*{-1mm}e \hspace*{3mm}
    \answerbox\hspace*{-1mm}f \hspace*{3mm}
}

\newcommand{\fiobox}{
\fbox{
  \begin{minipage}{42em}
    Имя, фамилия и номер группы:\vspace*{3ex}\par
    \noindent\dotfill\vspace{2mm}
  \end{minipage}
}
}

% делаем короче интервал в списках
\setlength{\itemsep}{0pt}
\setlength{\parskip}{0pt}
\setlength{\parsep}{0pt}

\begin{document}
\fiobox

\begin{multicols}{2}
\begin{enumerate}
    \item \answerline
    \item \answerline
    \item \answerline
    \item \answerline
    \item \answerline
    \item \answerline
    \item \answerline
    \item \answerline
    \item \answerline
    \item \answerline
    \item \answerline
    \item \answerline
    \item \answerline
    \item \answerline
    \item \answerline
    \item \answerline
    \item \answerline
    \item \answerline
    \item \answerline
    \item \answerline

\end{enumerate}
\end{multicols}

\newpage
Удачи!
\newpage

\fiobox

\begin{enumerate}

\begin{question}
Рассмотрим бесконечно повторяемую классическую дилемму заключенного с
дисконт-фактором \(\delta\).

Сколько существует различных равновесий Нэша, совершенных в подыграх,
при \(\delta \to 1\)?
\begin{answerlist}
  \item 2
  \item 1
  \item бесконечно много
  \item нет верного ответа
  \item 4
  \item 3
\end{answerlist}
\end{question}




\begin{question}
Выберите верное утверждение о SPNE (равновесии Нэша, совершенном в
подыграх) и NE (равновесии Нэша).
\begin{answerlist}
  \item Если в игре есть подыгры помимо игры в целом, то количество SPNE строго
больше количества NE.
  \item нет верного ответа
  \item Если в игре нет других подыгр, кроме игры в целом, то количество NE
меньше количества SPNE.
  \item Если в игре нет других подыгр, кроме игры в целом, то каждое NE является
SPNE.
  \item Если в игре есть подыгры помимо игры в целом, то количество SPNE строго
меньше количества NE.
  \item Если в игре нет других подыгр, кроме игры в целом, то количество NE
больше количества SPNE.
\end{answerlist}
\end{question}




\begin{question}
Рассмотрим дерево игры с совершенной информацией. Первый игрок делает
ход в двух узлах дерева, второй игрок делает ход в других двух узлах. В
каждом узле у каждого игрока 7 вариантов хода. Узлы второго игрока лежат
в одном информационном множестве.

Укажите сумму количества чистых стратегий первого игрока и количества
чистых стратегий второго игрока.
\begin{answerlist}
  \item 77
  \item 70
  \item 42
  \item нет верного ответа
  \item 56
  \item 49
\end{answerlist}
\end{question}




\begin{question}
Рассмотрим одновременную игру двух игроков. У первого игрока 7 чистых
стратегий, у второго --- 6 чистых стратегий.

Сколько всего есть смешанных стратегий у первого игрока?
\begin{answerlist}
  \item 7
  \item 14
  \item нет верного ответа
  \item 6
  \item 8
  \item 42
\end{answerlist}
\end{question}




\begin{question}
Рассмотрим одновременную игру двух игроков, у каждого из которых две чистых стратегии.

Сколько возможно равновесий Нэша в чистых стратегиях?
\begin{answerlist}
  \item \(\{0, 1\}\)
  \item \(\{0, 1, 3 \}\)
  \item \(\{0, 1, 2, 3\}\)
  \item \(\{0, 1, 2, 4\}\)
  \item \(\{0, 1, 2, 3, 4\}\)
  \item нет верного ответа
\end{answerlist}
\end{question}



\setzerocols

\begin{question}
Рассмотрим бесконечно повторяемую дилемму заключенного с
дисконт-фактором \(\delta\). \[
\begin{matrix}
   & c & d \\
c  & (9, 9) & (5, 13) \\
d  & (13, 5) & (6, 6) \\
\end{matrix}
\]

При каком наименьшем \(\delta\) пара стратегий жёсткого переключения
(grim trigger) будет равновесием Нэша, совершенным в подыграх?

Ответы указаны с точностью до двух знаков после запятой.
\begin{answerlist}
  \item 0.33
  \item нет верного ответа
  \item 0.29
  \item 0.57
  \item 0.4
  \item 0.25
\end{answerlist}
\end{question}




\begin{question}
Саша выбирает действительное число \(s\), затем Тоша выбирает
действительное число \(t\), зная выбор Саши. Выигрыш Саши равен
\(u_S = -s^2 + 8t\), выигрыш Тоши равен \(u_T = -t^2 + 5 st\).

Какое число выберет Саша в равновесии Нэша, совершенном в подыграх?
\begin{answerlist}
  \item 6.67
  \item 5
  \item 8
  \item 10
  \item нет верного ответа
  \item 13.33
\end{answerlist}
\end{question}



\setncols{3}

\begin{question}
Рассмотрим одновременную игру в которую играют 5 игроков, у каждого из которых конечное число стратегий.

Что может произойти с количеством равновесий Нэша в чистых стратегиях, \(n_{NE}\), и количеством Парето-оптимальных исходов в чистых стратегиях, \(n_{PO}\),
при увеличении выигрыша первого игрока на 5 во всех исходах?
\begin{answerlist}
  \item \(n_{NE}\) не изменится, \(n_{PO}\) может измениться в любую сторону
  \item \(n_{NE}\) может измениться в любую сторону, \(n_{PO}\) может только вырасти
  \item нет верного ответа
  \item \(n_{NE}\) может только вырасти, \(n_{PO}\) может только упасть
  \item \(n_{NE}\) может только вырасти, \(n_{PO}\) не изменится
  \item \(n_{NE}\) и \(n_{PO}\) могут измениться в любую сторону
\end{answerlist}
\end{question}



\setzerocols

\begin{question}
Рассмотрим одновременную игру с матрицей \[
\begin{matrix}
   & e & f \\
a  & (4, 6) & (2, 1) \\
b  & (2, 1) & (4, 4) \\
c  & (2, 1) & (2, 1). \\
\end{matrix}
\]

Найдите вероятность, с которой первый игрок использует стратегию «a» в
смешанном равновесии Нэша.

Ответы указаны с точностью до двух знаков после запятой.
\begin{answerlist}
  \item 0.3
  \item 0.27
  \item нет верного ответа
  \item 0.38
  \item 0.33
  \item 0.25
\end{answerlist}
\end{question}



\setncols{3}

\begin{question}
Рассмотрим одновременную игру двух игроков. У первого игрока 6 чистых
стратегий, у второго --- 7 чистых стратегий.

Сколько всего есть смешанных стратегий у первого игрока?
\begin{answerlist}
  \item 5
  \item 7
  \item нет верного ответа
  \item 6
  \item 42
  \item 12
\end{answerlist}
\end{question}



\setzerocols

\begin{question}
Рассмотрим одновременную игру двух игроков, у каждого из которых две
чистых стратегии.

Сколько возможно равновесий Нэша в смешанных стратегиях?
\begin{answerlist}
  \item \(\{0, 1, 2, 3, \infty \}\)
  \item \(\{0, 1, \infty \}\)
  \item \(\{1, 2, \infty \}\)
  \item нет верного ответа
  \item \(\{0, 1\}\)
  \item \(\{1, 2, 3, \infty \}\)
\end{answerlist}
\end{question}



\setncols{3}

\begin{question}
Рассмотрим одновременную игру с матрицей \[
\begin{matrix}
   & e & f \\
a  & (3, 4) & (5, 4) \\
b  & (6, 2) & (3, 3) \\ 
c  & (5, 0) & (4, 2). \\ 
\end{matrix}
\]

Найдите множество наилучших ответов первого игрока на смешанную
стратегию второго \(s_2 = 0.3e + 0.7f\).
\begin{answerlist}
  \item \(\{a\}\)
  \item \(\{b, c\}\)
  \item \(\{a, b\}\)
  \item \(\{b\}\)
  \item нет верного ответа
  \item \(\{c\}\)
\end{answerlist}
\end{question}




\begin{question}
Рассмотрим дерево игры с совершенной информацией. Первый игрок делает
ход в двух узлах дерева, второй игрок делает ход в других двух узлах. В
каждом узле у каждого игрока 8 вариантов хода. Узлы второго игрока лежат
в одном информационном множестве.

Укажите количество вероятностей, необходимых для описания поведенческой
стратегии первого игрока.

«Последнюю» вероятность считать не нужно, так как она определяется
ограничением на сумму вероятностей.
\begin{answerlist}
  \item 15
  \item 13
  \item 17
  \item 16
  \item нет верного ответа
  \item 14
\end{answerlist}
\end{question}




\begin{question}
Выберите верное утверждение о произвольной кооперативной игре в
коалиционной форме для конечного числа игроков.
\begin{answerlist}
  \item Ядро может быть пустым, но если оно непусто, то вектор Шепли лежит в
ядре.
  \item Ядро всегда непусто, вектор Шепли обязан лежать в ядре.
  \item Ядро всегда непусто, вектор Шепли может не лежать в ядре.
  \item Вектор Шепли не существует, если ядро пусто.
  \item Вектор Шепли всегда существует и единственный.
  \item нет верного ответа
\end{answerlist}
\end{question}




\begin{question}
Рассмотрим дерево игры с совершенной информацией. Первый игрок делает
ход в двух узлах дерева, второй игрок делает ход в других двух узлах. В
каждом узле у каждого игрока 7 вариантов хода. Узлы второго игрока лежат
в одном информационном множестве.

Укажите сумму количества чистых стратегий первого игрока и количества
чистых стратегий второго игрока.
\begin{answerlist}
  \item 84
  \item 70
  \item 42
  \item 56
  \item нет верного ответа
  \item 91
\end{answerlist}
\end{question}




\begin{question}
Саша и Тоша одновременно выбирают действительные числа \(s\) и \(t\).
Полезность Тоши равна \(u_T = -t^2 + 14st\). Саша может равновероятно
быть в хорошем или плохом настроении. В хорошем настроении полезность
Саши равна \(u_S = -s^2 + 2s\), в плохом --- \(u_S = -s^2 - 2st\).

Саша чуствует своё настроение, а Тоша не чуствует настроение Саши.

Какое \(t\) выбирает Тоша в равновесии Байеса-Нэша?
\begin{answerlist}
  \item 2.2
  \item 3.08
  \item 1.32
  \item нет верного ответа
  \item 0.44
  \item 0.88
\end{answerlist}
\end{question}




\begin{question}
Саша выбирает действительное число \(s\), затем Тоша выбирает
действительное число \(t\), зная выбор Саши. Выигрыш Саши равен
\(u_S = -s^2 + 8t\), выигрыш Тоши равен \(u_T = -t^2 + 7 st\).

Какое число выберет Саша в равновесии Нэша, совершенном в подыграх?
\begin{answerlist}
  \item 14
  \item 18.67
  \item 7
  \item нет верного ответа
  \item 9.33
  \item 11.2
\end{answerlist}
\end{question}




\begin{question}
Лиса перевела игру двух игроков из развёрнутой формы в биматричную и передала биматричную форму Белке.

Выберите верное утверждение.
\begin{answerlist}
  \item Белка сможет найти в исходной игре все равновесия Нэша совершенные в подыграх
  \item нет верного ответа
  \item Белка сможет найти все равновесия Нэша в исходной игре, но не обязательно сможет понять, какие из них совершенны в подыграх
  \item Белка сможет только посчитать количество равновесий Нэша в исходной игре, но не сможет их найти
  \item Белка сможет посчитать количество подыгр в исходной игре
  \item Белка сможет посчитать в исходной игре количество равновесий Нэша совершенных в подыграх
\end{answerlist}
\end{question}




\begin{question}
За день Ыуы может откопать 9 кореньев, а Уыу --- 11 килограмм. Работая
вместе они откопали за день 33 килограмм кореньев. Сколько килограмм
должен получить Ыуы в векторе Шепли?

Ответы округлены с точностью до двух знаков после запятой.
\begin{answerlist}
  \item 9
  \item 16.15
  \item 14.85
  \item 15.5
  \item 22
  \item нет верного ответа
\end{answerlist}
\end{question}




\begin{question}
Рассмотрим одновременную игру с матрицей \[
\begin{matrix}
   & e & f \\
a  & (5, 10) & (3, 3) \\
b  & (3, 3) & (5, 6) \\
c  & (3, 3) & (3, 3). \\
\end{matrix}
\]

Найдите вероятность, с которой первый игрок использует стратегию «a» в
смешанном равновесии Нэша.

Ответы указаны с точностью до двух знаков после запятой.
\begin{answerlist}
  \item 0.3
  \item 0.25
  \item 0.21
  \item 0.23
  \item 0.27
  \item нет верного ответа
\end{answerlist}
\end{question}



\end{enumerate}
\end{document}
