
\begin{question}
Рассмотрим дерево игры с совершенной информацией. Первый игрок делает
ход в двух узлах дерева, второй игрок делает ход в других двух узлах. В
каждом узле у каждого игрока 7 вариантов хода. Узлы второго игрока лежат
в одном информационном множестве.

Укажите количество вероятностей, необходимых для описания поведенческой
стратегии первого игрока.

«Последнюю» вероятность считать не нужно, так как она определяется
ограничением на сумму вероятностей.
\begin{answerlist}
  \item нет верного ответа
  \item 12
  \item 17
  \item 10
  \item 16
  \item 14
\end{answerlist}
\end{question}


