
\begin{question}
Рассмотрим одновременную игру с матрицей \[
\begin{matrix}
   & e & f \\
a  & (5, 10) & (3, 3) \\
b  & (3, 3) & (5, 6) \\
c  & (3, 3) & (3, 3). \\
\end{matrix}
\]

Найдите вероятность, с которой первый игрок использует стратегию «a» в
смешанном равновесии Нэша.

Ответы указаны с точностью до двух знаков после запятой.
\begin{answerlist}
  \item 0.3
  \item 0.25
  \item 0.21
  \item 0.23
  \item 0.27
  \item нет верного ответа
\end{answerlist}
\end{question}


