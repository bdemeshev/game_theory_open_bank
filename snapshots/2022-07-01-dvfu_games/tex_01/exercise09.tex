
\begin{question}
Рассмотрим одновременную игру с матрицей \[
\begin{matrix}
   & e & f \\
a  & (4, 6) & (2, 1) \\
b  & (2, 1) & (4, 4) \\
c  & (2, 1) & (2, 1). \\
\end{matrix}
\]

Найдите вероятность, с которой первый игрок использует стратегию «a» в
смешанном равновесии Нэша.

Ответы указаны с точностью до двух знаков после запятой.
\begin{answerlist}
  \item 0.3
  \item 0.27
  \item нет верного ответа
  \item 0.38
  \item 0.33
  \item 0.25
\end{answerlist}
\end{question}


