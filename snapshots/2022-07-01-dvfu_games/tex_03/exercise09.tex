
\begin{question}
У Саши три чистых стратегии, \(a\), \(b\) и \(c\). В единственном
смешанном равновесии Нэша она выбирает \(a\) с вероятностью 0.2, \(b\)
--- с вероятностью 0.6.

Что можно утверждать об ожидаемых выигрышах Саши от выбора этих
стратегий при фиксированных стратегиях остальных игроков?
\begin{answerlist}
  \item \(u(c) > u(a)\)
  \item \(u(a) > u(b)\)
  \item \(u(a) < u(b)\)
  \item \(u(c) < u(a)\)
  \item \(u(c) > u(b)\)
  \item нет верного ответа
\end{answerlist}
\end{question}


