
\begin{question}
Саша и Тоша одновременно выбирают действительные числа \(s\) и \(t\).
Полезность Тоши равна \(u_T = -t^2 + 18st\).
Саша может равновероятно быть в хорошем или плохом настроении.
В хорошем настроении полезность Саши равна \(u_S = -s^2 + 2s\), в плохом --- \(u_S = -s^2 - 2st\).

Саша чуствует своё настроение, а Тоша не чуствует настроение Саши.

Какое \(t\) выбирает Тоша в равновесии Байеса-Нэша?
\begin{answerlist}
  \item 0.45
  \item 0.9
  \item 1.35
  \item 3.15
  \item 2.25
  \item нет верного ответа
\end{answerlist}
\end{question}


